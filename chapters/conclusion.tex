The results presented in chapter \ref{ai}, shows us that the real world problems that are data intensive and memory consuming, can be simulated with the help of parallel processing. They can be simulated with a good speedup and can utilize the hardware capabilities quiet effectively as shown by the results of efficiency. With harnessing the computing power and with larger pool of memory of many nodes connected with fast interconnects, we can simulate large problem in the fields of computational fluid dynamics, medical simulations, big data and many more.  

\vspace{10pt}
\hspace{-18pt}We also have proposed a alternative methodology for parallelization in chapter \ref{am}. The implementation has not been done yet for alternative methodology and that remains the work for future projects. 

\vspace{10pt}
\hspace{-18pt}The code following the methodology presented in chapter \ref{m} is available on Github but currently it is private. This code can be accessed on this \href{}{link} after the paper \cite{arpit} gets accepted for publication.