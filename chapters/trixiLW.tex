\documentclass[12pt]{article}
\usepackage[a4paper, width=150mm,top=25mm,bottom=25mm]{geometry}
\usepackage{amsmath}
\usepackage{amsfonts}
\usepackage{hyperref}
\usepackage{minted}
\usepackage{listings}

\lstset{basicstyle=\ttfamily}

\begin{document}
\section{Simulation of Conservation laws}
\subsection{Trixi.jl}
\href{https://github.com/trixi-framework/Trixi.jl}{\lstinline[language=Python]|Trixi.jl|} is a numerical simulation framework for conservation laws written in 
\lstinline[language=Python]|Julia|.
\subsection{Features of {\ttfamily Trixi.jl}}

\lstinline[language=Python]|Trixi.jl| package can be used for conservation laws in $1$D, $2$D and $3$D dimensions with the following features:
\begin{itemize}
    \item Support cartesian and curvilinear meshes.
    \item Structured and Unstructured meshes.
    \item AMR and Shock capturing.
    \item High-order accuracy in space and time.
    \item Discontinuous Galerkin methods.
    \item CFL-based and error-based time step control.
    \item Periodic and weakly-enforced boundary conditions.
\end{itemize}
{\ttfamily Trixi.jl} has multiple governing equations such as {\ttfamily Compressible Euler, Compressible Navier-Stokes, Shallow water equations} etc.
The code written in this package also have support for {\ttfamily shared memory} parallelization via {\ttfamily multi-threading} and {\ttfamily multi-node} parallelization
via  {\ttfamily Message Passing Interface (MPI)}. It also provides visualization and post-processing support for the results.

It uses $Runge-Kutta$ methods for discretization in time with other high-orders for space discretization.

\subsection{{\ttfamily TrixiLW.jl}}
In {\ttfamily TrixiLW}, we have extended the {\ttfamily Trixi} package to use $Lax-Wendroff~flux\\reconstruction$ method for time discretization instead of $RK$ methods. 

\subsubsection{Features of {\ttfamily TrixiLW}}
{\ttfamily TrixiLW} can be used for conservation laws in $2$D with the following features:
\begin{itemize}
    \item Support for cartesian and curvilinear meshes.
    \item Structured and Unstructured meshes.
    \item High-order accuracy in space and time.
    \item Discontinuous Galerkin methods.
    \item CFL-based and error-based time step control.
    \item Periodic and non-periodic boundary conditions.
    \item AMR and Shock capturing
\end{itemize}
This package also supports {\ttfamily shared memory} parallelization via {\ttfamily multi-threading} and {\ttfamily multi-node} parallelization via {\ttfamily Message Passing
Interface (MPI)}. For visualization and post-processing, we use the tools of {\ttfamily Trixi} itself.

\subsubsection{{\ttfamily Multi-node} parallelization of {\ttfamily TrixiLW}}
{\ttfamily Multi-node} parallelization implies that we can use the computing units that not present on a single node on which the program is running. This can help us encounter
the issue of insufficient memory on a single node. We, now, can use memory of other nodes which are connected to each other via channel-based fabric such as {\ttfamily InfiniBand}
that facilitates high-speed communication between interconnected nodes. 

Although, using {\ttfamily multi-node} parallelization can also creates a communication overhead for the running program yet it is used since their is a nice trade-off between
the overhead created due to communication and other aspects such as memory, computing power, data storage etc. 


\end{document}