In this project, I have added support for multi-node parallelization in {\ttfamily TrixiLW.jl}. The code that has been written for this is available on Github and the link is provided in chapter \ref{c}. The code has been tested for two problem as mentioned in chapter \ref{ai} and respective initial conditions and results that have been generated by the code are shown there. These presented results have been verified with the results of serial version of the code. The serial version of the code and its respective results can be found in this paper \cite{arpit}. The comprehensive results for parallel code are shown in analysis section \ref{ai} \\ \\
Scaling tests for the parallel code have also been performed for both the problem and the results have been shown in the section \ref{st}. The results of scaling test shows that the code has performed well with minimizing the overhead due to the MPI communication which is the major challenge of using parallel algorithms. \\ \\
The efficiency results shows that the code is able to use the underline hardware effectively for communication which is a good sign for a parallel algorithm as these days we are seeing rapid development in networking and computing hardware.\\ \\
Animation of the solution for \href{https://youtu.be/Q1fwe4AS2fk}{isentropic vortex problem} and \href{https://youtu.be/06WHI8WRZzM}{kelvin-helmholtz instability.}